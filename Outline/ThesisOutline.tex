\documentclass{article}
\usepackage[utf8]{inputenc}
\usepackage{outline}

\title{Examining the Security of Local Inter-Process Communication}
\author{Brendan Leech}
\date{Advised by Prof. Peter C. Johnson}

\begin{document}

\maketitle

\begin{outline}
    \item {Abstract}
    \begin{itemize}
        \item {Many applications have more than one process running}
        \item {They communicate over local IPC (completely within the machine)}
        \item {This communication has been shown to have vulnerabilities}
        \item {This thesis will look at commonly-used applications to find what, if any, security issues can be found.}
    \end{itemize}
    \item {Introduction}
    \begin{itemize}
        \item {Many modern applications are split into multiple processes}
        \item {This helps to separate functionality, so that each individual process can do its specific job}
        \item {However, for these processes to work together, they need to be able to communicate}
        \item {This communication has been shown to be insecure}
        \item {Since the communication never leaves the physical machine, an attacker would need to have access to the computer}
        \item {Many computers have guest accounts, and there are ways to keep programs running after a user has logged out (nohup and fast-user switching)}
        \item {Therefore, an attacker could start a malicious program and logout, without the victim having any idea that their machine has been compromised}
        \item Plan of the Thesis
        \begin{itemize}
            \item Create survey to find commonly-used applications and their local IPC footprints
            \item Using the results, find ``interesting" programs to study
            \item Use fuzzing software and tools to examine the communication to try and find vulnerabilities
            \item Fuzzing can find crashes
            \item Studying the communication can find ways to steal information or hijack execution
        \end{itemize}
    \end{itemize}
    \item {Background}
    \begin{outline}
        \item {Inter-Process Communication}
        \begin{outline}
            \item What is IPC
            \begin{itemize}
                \item IPC is the way that any two programs communicate with each other
                \item Broadly speaking, everything from email, to webpages, to Spotify, uses IPC
                \item The most common form people know is internet communication, which allows separate computers to connect
            \end{itemize}
            \item Local IPC
            \begin{itemize}
                \item This communication occurs entirely within one computer
                \item Since it stays within the computer, many times the overhead required to communicate between computers can be avoided
                \item This can make the communication significantly faster
                \item Because applications are split into multiple processes, they also much communicate sensitive data
                \item Since its within one machine, some security experts believe it is not worth trying to encrypt (CITATION NEEDED)
                \item However, many programs make an effort to encrypt it, but do a worse job than with their networked communication
            \end{itemize}
            \item Forms of Local IPC
            \begin{outline}
                \item Communication to localhost
                \begin{outline}
                    \item How does this work (explain full network stack)
                    \item Why is localhost useful
                \end{outline}
                \item UNIX Domain Sockets
                \begin{outline}
                    \item How does this work
                    \item How is this different from localhost
                    \item What are the tradeoffs between localhost and these
                \end{outline}
                \item Named Pipes
                \begin{outline}
                    \item How does these work
                    \item What are the benefits over localhost/UNIX domain sockets
                    \item Differences between these and anonymous pipes
                \end{outline}
            \end{outline}
            \item Local IPC Vulnerabilities
            \begin{itemize}
                \item Discuss how local IPC is sometimes ignored, or at the very least, secured to a lesser extent than networked communication
                \item Vulnerabilities in named pipes
                \item Poor encryption that allows listeners to steal information
            \end{itemize}
        \end{outline}
        \item Fuzzing
        \begin{outline}
            \item What is fuzzing
            \begin{itemize}
                \item Sending random or semi-random input to a process to try and induce a crash
                \item Try to cause undesired behavior by the program
                \item This can be done by the authors, or by third parties
                \item Why either side would want to do it
            \end{itemize}
            \item The fuzzers I am using
            \begin{itemize}
                \item Radamsa
                \item Sulley
                \item afl
                \item What makes them different from each other
            \end{itemize}
        \end{outline}
        \item How to see local IPC
        \begin{outline}
            \item For localhost - Wireshark
            \item For UNIX domain sockets - sniff the packets as they are sent or modify kernel functions to send messages to another location as well
            \item For named pipes - join the pipe as a reader
        \end{outline}
    \end{outline}
    \item Results
    \begin{outline}
        \item Application 1
        \begin{itemize}
            \item How does IPC work for this application?
            \item What did we learn from studying communication?
            \item What did we learn from fuzzing?
        \end{itemize}
        \item Application 2
        \begin{itemize}
            \item How does IPC work for this application?
            \item What did we learn from studying communication?
            \item What did we learn from fuzzing?
        \end{itemize}
        \item Application 3
        \begin{itemize}
            \item How does IPC work for this application?
            \item What did we learn from studying communication?
            \item What did we learn from fuzzing?
        \end{itemize}
    \end{outline}
    \item Future Work
    \begin{itemize}
        \item What can be done to avoid these security issues in the future?
        \item Would these changes be difficult to complete?
    \end{itemize}
    \item Discussion
    \begin{itemize}
        \item We have found X bugs/vulnerabilities in these applications that use local IPC
        \item These bugs are due to: careless programming and unknown attack vectors
        \item They can be fixed in the future with two-way authentication or an air gap
    \end{itemize}
    \item Acknowledgements
\end{outline}

\nocite{*}
\bibliographystyle{plainurl}
\bibliography{references}

\end{document}
